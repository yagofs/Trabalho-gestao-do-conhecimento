\begin{resumo}
A Gestão do Conhecimento emerge como uma disciplina estratégica fundamental para organizações que buscam otimizar seus recursos intelectuais e sustentar a inovação em um cenário global dinâmico e competitivo. Este trabalho explora a fundamentação teórica e a contextualização das Bases de Conhecimento no âmbito da Gestão do Conhecimento. São abordados conceitos essenciais como a distinção entre conhecimento tácito e explícito, bem como a evolução histórica das Bases de Conhecimento desde suas origens na Inteligência Artificial na década de 1970 até sua consolidação como ferramentas estratégicas de gestão organizacional. O documento apresenta a contextualização das Bases de Conhecimento nos modelos clássicos da área, com destaque para o modelo SECI de Nonaka e Takeuchi e a caracterização dos Sistemas de Gestão do Conhecimento por Alavi e Leidner. Analisa-se o funcionamento das Bases de Conhecimento como ecossistemas que integram pessoas, processos e tecnologia, detalhando elementos como organização do conteúdo, indexação, mecanismos de busca e processos de atualização. Conclui-se que as Bases de Conhecimento são infraestrutura crítica que sustenta a memória organizacional e impulsiona a inovação, sendo a sinergia entre tecnologia, processos e engajamento humano o fator determinante para seu sucesso estratégico.

\textbf{Palavras-chave}: Gestão do conhecimento. Base de conhecimento. Conhecimento organizacional. Sistemas de gestão do conhecimento. Capital intelectual.
\end{resumo}