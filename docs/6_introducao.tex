\chapter{Introdução}

A Gestão do Conhecimento (GC) emerge como uma disciplina estratégica fundamental para organizações que buscam otimizar seus recursos intelectuais e sustentar a inovação em um cenário global dinâmico e competitivo. Conforme delineado por Davenport e Prusak \cite{davenport1998}, a GC abrange um conjunto sistemático de processos voltados para a criação, organização, disseminação e aplicação do conhecimento, visando aprimorar o desempenho organizacional. Dentro deste arcabouço, as Bases de Conhecimento (BCs) desempenham um papel central, atuando como repositórios estruturados que facilitam o armazenamento, a recuperação e a reutilização do capital intelectual de uma organização.

As BCs são mais do que meros depósitos de informações; elas representam a memória organizacional, um componente crítico para a continuidade e a evolução do saber coletivo. Alavi e Leidner \cite{alavi2001} destacam que a implementação eficaz de BCs é crucial para mitigar a perda de conhecimento institucional e para acelerar a curva de aprendizado dos colaboradores.

\section{Relevância do Tema}

A relevância deste tema é acentuada no ambiente organizacional contemporâneo, onde a capacidade de acessar e aplicar o conhecimento de forma ágil e eficiente se traduz diretamente em vantagem competitiva e resiliência estratégica.

\section{Objetivo do Documento}

Este documento visa explorar a fundamentação teórica e a contextualização das Bases de Conhecimento no âmbito da Gestão do Conhecimento, abordando sua evolução histórica, seu funcionamento e sua contribuição para os processos de criação, armazenamento, compartilhamento e reutilização do conhecimento organizacional.
