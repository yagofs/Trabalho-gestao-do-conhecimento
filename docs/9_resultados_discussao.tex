\chapter{Contextualização na Gestão do Conhecimento}

No âmbito da Gestão do Conhecimento (GC), as Bases de Conhecimento (BCs) são elementos intrínsecos e fundamentais para a operacionalização dos modelos e teorias clássicas da área. Elas se inserem como ferramentas essenciais para a materialização e o suporte dos processos de conhecimento organizacional.

\section{O Modelo SECI}

Um dos modelos mais influentes na GC é o de criação de conhecimento organizacional proposto por Nonaka e Takeuchi \cite{nonaka1997}, conhecido como modelo SECI (Socialização, Externalização, Combinação e Internalização). As BCs desempenham um papel crucial em duas das quatro fases deste modelo: a \textbf{Externalização} e a \textbf{Combinação}.

Na fase de Externalização, o conhecimento tácito, que reside na mente dos indivíduos, é convertido em conhecimento explícito, formalizado e codificado. As BCs servem como o repositório onde esse conhecimento explícito é armazenado, tornando-o acessível a outros membros da organização.

Na fase de Combinação, diferentes conhecimentos explícitos são integrados e sistematizados, resultando em novos conhecimentos explícitos. As BCs facilitam essa integração ao proverem uma plataforma para a organização, indexação e recuperação de informações e documentos diversos, permitindo que novos insights e soluções sejam gerados a partir da articulação de conhecimentos existentes.

\section{Processos de Conhecimento Organizacional}

Alavi e Leidner \cite{alavi2001}, em sua revisão sobre Sistemas de Gestão do Conhecimento (KMS), destacam que tais sistemas, nos quais as BCs estão inseridas, apoiam os processos de criação, armazenamento/recuperação, transferência e aplicação do conhecimento.

As BCs são, portanto, o pilar central do processo de \textbf{armazenamento e recuperação} do conhecimento organizacional. Elas garantem que o conhecimento gerado não se perca, seja preservado de forma estruturada e possa ser acessado de maneira eficiente quando necessário.

Ao fazer isso, as BCs contribuem diretamente para a \textbf{transferência} do conhecimento, pois disponibilizam informações e saberes codificados para toda a organização, e indiretamente para a \textbf{criação} (ao fornecer subsídios para novos desenvolvimentos) e \textbf{aplicação} (ao oferecer suporte para a tomada de decisões e resolução de problemas).

\section{Reutilização do Conhecimento}

Além disso, as BCs contribuem significativamente para a \textbf{reutilização do conhecimento organizacional}. Ao centralizar e estruturar informações, elas evitam a duplicação de esforços, reduzem o tempo gasto na busca por soluções e promovem a padronização de processos e melhores práticas.

A capacidade de uma organização de aprender com suas experiências passadas e de aplicar esse aprendizado em situações futuras é amplamente potencializada pela existência de BCs bem gerenciadas. Autores como Terra \cite{terra2005} reforçam a importância da infraestrutura tecnológica, onde as BCs se encaixam, como um dos pilares para uma gestão do conhecimento eficaz, ao lado de aspectos culturais e processuais.

Assim, as Bases de Conhecimento não são apenas ferramentas tecnológicas, mas componentes estratégicos que integram pessoas, processos e tecnologia para otimizar o fluxo e o valor do conhecimento dentro das organizações.

\chapter{Funcionamento das Bases de Conhecimento}

O funcionamento de uma Base de Conhecimento (BC) transcende a mera implementação de um sistema tecnológico; ele se configura como um ecossistema complexo que integra pessoas, processos e tecnologia para gerenciar o conhecimento organizacional de forma eficaz. Conceitualmente, uma BC atua como um repositório centralizado e estruturado de informações e conhecimentos, projetado para facilitar sua captura, organização, busca, recuperação e disseminação.

\section{Gestão dos Tipos de Conhecimento}

Um dos desafios centrais no funcionamento das BCs reside na gestão dos diferentes tipos de conhecimento: o \textbf{conhecimento tácito} e o \textbf{conhecimento explícito} \cite{nonaka1997}. O conhecimento explícito, por ser formalizado e codificável (documentos, manuais, relatórios), é diretamente armazenado nas BCs.

Já o conhecimento tácito, que é pessoal, experiencial e difícil de ser articulado, não pode ser diretamente armazenado. No entanto, as BCs desempenham um papel crucial na sua \textbf{externalização}, ao fornecerem ferramentas e processos para que os indivíduos transformem seu conhecimento tácito em formas explícitas, como artigos, tutoriais ou melhores práticas documentadas.

Além disso, as BCs podem indiretamente facilitar o acesso ao conhecimento tácito ao indexar os especialistas (Subject Matter Experts - SMEs) da organização, permitindo que os usuários identifiquem e contatem as fontes de conhecimento tácito.

\section{Estrutura Básica de uma Base de Conhecimento}

A estrutura básica de uma Base de Conhecimento é composta por diversos elementos interligados que garantem sua funcionalidade:

\begin{itemize}
    \item \textbf{Organização do Conteúdo:} Envolve a categorização e classificação do conhecimento utilizando taxonomias, ontologias e estruturas hierárquicas. Essa organização sistemática é fundamental para a navegabilidade e a compreensão do vasto volume de informações.
    
    \item \textbf{Indexação:} A aplicação de metadados e palavras-chave aos itens de conhecimento é vital para otimizar os mecanismos de busca. Uma indexação robusta assegura que o conteúdo seja facilmente recuperável.
    
    \item \textbf{Mecanismos de Busca:} Motores de busca avançados, muitas vezes com capacidades de busca semântica, são componentes essenciais. Eles permitem que os usuários localizem informações relevantes de maneira rápida e precisa.
    
    \item \textbf{Atualização e Manutenção:} Uma BC eficaz não é estática. Ela requer um ciclo de vida de conhecimento bem definido, que inclui processos contínuos de criação, revisão, validação, atualização e arquivamento de conteúdo.
\end{itemize}

\section{O Papel das Pessoas}

O \textbf{papel das pessoas} é insubstituível no funcionamento das BCs. Além dos usuários finais que consomem o conhecimento, há os criadores de conteúdo, os especialistas que validam as informações e os gestores de conhecimento (Knowledge Managers) que supervisionam a curadoria e a evolução da BC.

A cultura organizacional de compartilhamento e colaboração, conforme enfatizado por Terra \cite{terra2005}, é um fator crítico de sucesso. Sem o engajamento das pessoas, a BC se torna um repositório subutilizado.

\section{Processos e Tecnologia}

Os \textbf{processos} definem como o conhecimento é capturado, validado, publicado e mantido. Isso inclui fluxos de trabalho para submissão de novos artigos, revisão por pares, aprovação e publicação. A padronização desses processos garante a qualidade e a consistência do conteúdo da BC.

A \textbf{tecnologia} fornece a infraestrutura para o funcionamento da BC, abrangendo desde plataformas de software (wikis corporativas, sistemas de gerenciamento de conteúdo, repositórios de documentos) até ferramentas de inteligência artificial que podem auxiliar na categorização automática, na personalização da busca e na identificação de lacunas de conhecimento.

A escolha da tecnologia deve estar alinhada com as necessidades e a cultura da organização, suportando os processos e incentivando a participação das pessoas.