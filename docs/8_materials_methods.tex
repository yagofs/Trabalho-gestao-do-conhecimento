\chapter{Histórico e Linha do Tempo das Bases de Conhecimento}

A concepção das Bases de Conhecimento (BCs) possui raízes que remontam à década de 1970, emergindo no contexto da Inteligência Artificial (IA) e do desenvolvimento dos Sistemas Especialistas. Inicialmente, o termo ``Base de Conhecimento'' era empregado para diferenciar o repositório de regras de inferência e heurísticas de uma ``base de dados'', que armazenava fatos brutos. Essa distinção inicial sublinhava a natureza interpretativa e processual do conhecimento contido nas BCs, em contraste com a natureza meramente factual dos dados.

\section{Evolução Tecnológica}

A evolução das BCs está intrinsecamente ligada ao avanço da Tecnologia da Informação (TI). Nos anos 1980 e início dos 1990, com a proliferação de sistemas de gerenciamento de bancos de dados (SGBDs) e o surgimento das primeiras redes corporativas, o foco começou a se deslocar para a organização e o acesso à informação de forma mais ampla.

\section{Consolidação da Gestão do Conhecimento}

Contudo, foi a partir da consolidação da Gestão do Conhecimento como disciplina, notadamente com as contribuições de Nonaka e Takeuchi em 1995 \cite{nonaka1997}, que o conceito de BCs ganhou uma nova dimensão no ambiente organizacional.

Nonaka e Takeuchi, com seu modelo de criação de conhecimento (SECI), destacaram a importância da conversão entre conhecimento tácito e explícito. As BCs, nesse cenário, tornaram-se ferramentas cruciais para a externalização do conhecimento tácito (transformando-o em explícito e codificável) e para a combinação de diferentes formas de conhecimento explícito.

\section{Evolução Arquitetural}

Paralelamente, a evolução da TI, de sistemas centralizados para arquiteturas distribuídas, intranets e, posteriormente, a internet, permitiu que as BCs transcendessem a função de meros arquivos digitais, transformando-se em plataformas dinâmicas e colaborativas para a gestão do saber organizacional.

\section{Sistemas de Gestão do Conhecimento}

Um marco conceitual significativo foi a definição de Sistemas de Gestão do Conhecimento (KMS) por Alavi e Leidner em 2001 \cite{alavi2001}. Eles caracterizaram os KMS como sistemas baseados em TI projetados para apoiar e aprimorar os processos de criação, armazenamento/recuperação, transferência e aplicação do conhecimento nas organizações.

Dentro dessa perspectiva, as BCs representam o componente central de armazenamento e recuperação, sendo a infraestrutura tecnológica que suporta a memória organizacional e facilita o acesso ao conhecimento codificado.

\section{Tendências Atuais}

A contínua evolução tecnológica, incluindo o desenvolvimento de ferramentas de busca semântica, inteligência artificial e plataformas de colaboração, tem impulsionado as BCs a se tornarem sistemas cada vez mais sofisticados e integrados aos fluxos de trabalho organizacionais.